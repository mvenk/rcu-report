\documentclass{article}
\begin{document}
\author {
  Rohit Kumar 
  \\ UCLA 
  \\ \textsl{rohitkk@cs.ucla.edu}
  \and
  Madhuri Venkatesh 
  \\ UCLA 
  \\ \textsl{madhuri@cs.ucla.edu}
  \and
  \textsl{Supervised By:}
  \\Eddie Kohler 
  \\ UCLA 
  \\ \textsl{kohler@cs.ucla.edu}
}
\title{Cache Conscious Radix Tree Design}
\maketitle
\newcounter{hyp-count}
\addtocounter{hyp-count}{1}
\newcommand{\hypothesis}[1]{\paragraph{Hypothesis \arabic{hyp-count}:}
  #1 \addtocounter{hyp-count}{1}}

\pagebreak
\section{ABSTRACT}
\section{INTRODUCTION}
\section{RELATED WORK}
\section{EVALUATION}

\subsection{Changing Branching Factors}

\hypothesis{
A routing table which is dominated by longer mask values will have to
go deeper into the Radix tree to obtain the key. For example, for a
/24 mask the original Radix design would need to traverse 5 levels (/8
+ /4 + /4 + /4 + /4). Instead if we had a branching factor of $2^{16}$
at the first level and $2^{8}$ at the subsequent level, for the same
/24 mask value, we will only need to traverse two levels of the
tree. Reducing the number of levels, by increasing the branching
factor,  will cause the number of pointers
traversed to decrease and hence a reduction in cache misses. 
} 

\subsection{Using a Custom Allocator}
\hypothesis{
  At each Radix node there is space allocated for ``n'' number of
  child nodes, where ``n'' is the number of buckets present at that
  level. Along with this some extra space is allocated for ``n-2''
  keys (integer values) -- which are referred to as ``superchildren.''
  The superchildren are accessed only during updates to the table and
  not during lookup. Thus when a Radix node is fetched into a cache
  line, some space within the cache line is occupied by the
  superchildren which are never touched during lookup. We reason that
  at a level if we allocate all child nodes contiguously, without any
  superchildren interleaved between them, we will be able to fit more
  child nodes into the cache and get fewer misses. A custom allocator
  is needed to achieve this.
  }

\subsection{Removing Derivable Fields}
\hypothesis{The Radix node consists of fields which are a function of
  the depth at which the Radix node is present. These fields are
  referred to as ``derivable fields.'' These fields contribute to the
  size of the Radix node and will determine how many Radix nodes can
  fit into a cache line. If we remove the derivable fields and instead
  compute the value of each derived field when needed we will have the
  same functionality with a Radix node which is smaller in size. A
  smaller size lets us fit more nodes into one cache line and thus
  lower of cache misses.}

\hypothesis{When we remove the derivable fields, we compute the values
for these fields each time they are needed. Instead of computing this
each time, we can have a static array of pre-computed values which is
used whenever the values are needed. This will save us from executing
the instructions at a cost of additional memory.}

\subsection{Prefetching Child Nodes}
\hypothesis{While doing the lookup we iteratively compute the next child node
  to traverse. At the end of one iteration we know the next child
  node. This child node can be prefetched into the cache using the
  software prefetch instruction. By prefetching the child node before
  it's accessed will reduce the number of cache misses.
  }

\subsection{Using Separate Next Hop Table}
\hypothesis{The key returned by lookup indexes into the whole routing
  table, which returns the next hop and port. A routing table
  contains many unique address, mask, next hop and port combinations,
  but only a few unique next hop, port combinations. As the result of
  a lookup only needs the next hop and port, we could have a
  separate table which maintains only those two values.
  As this table will be smaller in size, as compared to the entire
  routing table, it will result in fewer cache misses.
}

\subsection{Skip Keys}
\hypothesis{ Within a Radix tree there could be internal nodes which
  have only one child, and that child has only one child, so on all
  the way to a leaf node. That is, there is only one path which can be
  followed out of that inner node. If we land on such a node, we
  already know which key will be returned, because there is only one
  path out. \\
  Instead of traversing the entire path, we can store the key
  within the inner node and skip the rest of the tree traversal. By
  avoiding this traversal we expect fewer cache misses.
}

\subsection{Detecting Null Children}

\hypothesis{ 
  A Radix child node has two fields: a key and a pointer
  to the child Radix node. Any given Radix node will have an array of
  child nodes contained within it. 

  When laid out in memory the array of child nodes look like this:\\


    [key : Radix *, key : Radix * ... ]\\

  
  We suggest an alterative layout of the child node fields, with an
  extra field called ``has\_child.'' The new layout has all the key
  values in the array bunched up together followed by all the Radix
  pointers. \\


    [key : has\_child, key : has\_child .... Radix *, Radix * ... ]

  
  The has\_child field is only 1-bit. We squeeze the has\_child field into
  the key field by making the key only 31-bits long. We use the
  has\_child field to detect whether a key has an associated child. If
  it doesn't, we avoid fetching the child pointer and would thus
  expect to see a lower number of child misses.
}



\section{CONCLUSION}
\section{FUTURE WORK}


\end{document}
