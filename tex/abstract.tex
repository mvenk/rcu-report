\begin{abstract}
High performance routing lookups are critically important for packet forwarding. Routers must perform a lookup for every arriving packet. As line rates exceed 40Gbps \cite{attlinkrate}, routers are required to perform 125 million lookups a second (assuming a minimum packet size of 40 bytes). Additionally, routers are required to perform sophisticated packet processing providing services which support security, video and mobility \cite{routebricks}. Software routers are more flexible and extensible for such processing when compared to hardware routers. However, software routers are an order of magnitude slower than their hardware counterparts \cite{cansoftwareroutersscale}. With multi-core systems becoming more popular and more prevalent, leveraging them is key to improving performance in software routers. However, obtaining a simultaneously correct and fast thread safe solution remains challenging. For example, conventional locking allows only a single thread to access a shared data structure at a time. Although this solution is easily correct, it does not take complete advantage of the available parallelism, and is evidently slower. Reader-writer locks, which allow readers to proceed concurrently with other readers, but not with updaters, are faster in comparison, yet slower than non-locking sequential performance.\\

We have described an approach for providing thread safety using \emph{Read-Copy Update} (RCU) \cite{readcopyupdate}. RCU is a synchronization technique which scales efficiently on read intensive workloads. It has very low read side overhead, allowing readers to proceed concurrently with other readers and updaters. A typical router workload is read intensive, with 0--5\% updates. For such a workload, using a commodity 8 core machine, we have verified that the performance of the RCU approach is near non-locking sequential performance, and is up to 27 times faster than a reader-writer lock.
\end{abstract}
